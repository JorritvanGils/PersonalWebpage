% set-up
\documentclass[a4paper, 10pt, final]{moderncv}
\usepackage[scale=0.75, top=3cm, bottom=3cm, left=2.5cm, right=2.5cm]{geometry}
\moderncvstyle{casual} 
\moderncvcolor{blue}
\usepackage[utf8]{inputenc}
\usepackage[english]{babel}
\nopagenumbers{} 

% personal
\firstname{Jorrit}
\familyname{van Gils}
\title{Machine Learning Engineer}
\mobile{(+31) 6 27 57 22 61}
\email{vangilsjorrit@gmail.com}
% \address{Jan Oomenstraat 5, Bavel, NL}
\homepage{www.jorritvangils.com/}
\photo[64pt][0.4pt]{Jorrit.jpg} 

\social[linkedin]{jorritvangils}
\social[github]{JorritvanGils}
% \social[twitter]{jorritvangils}

\begin{document}

\makecvtitle

\section{Summary}
% I’m a Computer Vision researcher at Wageningen University, advised by dr. Gert Kootstra dr. Gert Kootstra (Computer vision) and dr.ir Patrick Jansen (Ecology). 
% My research focusses on extracting wildlife behaviour from camera trap images using machine and deep learning algorithms. 
% Extensive experience in Python, JavaScript, Vue.js, and TensorFlow throught my role at object detection platform BOX21 and through annual 
% participation in Kaggle-hosted classification competitions.

Computer Vision and Large Language Models Engineer with five years of production experience building computer vision systems at BOX21 (2021–2024) and contributing to distributed AI on Bittensor (2024–2026). Previously researched wildlife behavior recognition from camera trap images at Wageningen University under supervision of dr. Gert Kootstra.

Continuously expanding my expertise through AI literature study and Kaggle competitions. A collaborative team player who values knowledge sharing and collective growth.

\section{Work Experience}
\cventry{2024--2026}{Open-Source Contributor}{Bittensor}{}{}{Developing federated learning AI training protocols at the AI-blockchain intersection}
\cventry{2021--2024}{Computer Vision Developer}{BOX21}{}{}{Developing object detection models and maintaining frontend-, backend systems}
\cventry{2023}{Fathomnet competition competitor}{Kaggle}{}{}{Detecting and classifying marine species in underwater images with Sean Nachtrab}
\cventry{2022}{Machine Learning Research Thesis}{Wageningen University}{}{}{Wildlife action recognition in camera-trap photographs using yolov5 and pose estimation}
\cventry{2022}{Satellite Image Classification Project}{}{}{}{Land-use classification using AlexNet on the UCM satellite dataset with Lars ter Kate}
\cventry{2022}{Competitor iWildCam competition}{Kaggle}{}{}{Counting the number of animals in a sequence of images}


\section{Computer skills}
\cvdoubleitem{Languages}{Python, Rust, R, Bash}
  {Frameworks}{PyTorch, TensorFlow, Keras, FastAi, Pandas, NumPy}
\cvdoubleitem{Computer Vision}{YOLOv5/8, Detectron2, OpenCV, mmdetection, DeepLabCut, SegmentAnything}
  {Foundations}{Mathematics, Linear Algebra, Statistics, AdamW, Nesterov, Cross Entropy, MSE}
\cvdoubleitem{NLP}{Transformers, Llama, TorchTitan, FinewebEdu, Tokenizers}
  {Tools}{HuggingFace, WandB, Docker, GitHub, Just, Linux, InfluxDB}
\cvdoubleitem{Dist. Training}{NCCL, FSDP, DDP, DiLoCo, LAMB, Quantization, Parallelism, AllReduce}
  {Web Development}{FastAPI, WebSockets, SQL, JavaScript, Vue.js, Nginx, RabbitMQ}

\section{Education}
\cventry{2019--2022}{MSc Forest and Nature Conservation}{Wageningen University, NL}{}{}{
    Minor in Artificial Intelligence: Programming in Python, Machine Learning, Deep Learning
}
% \cventry{2016--2019}{BSc Secondary School Teacher Biology}{Fontys Tilburg, NL}{}{}{}
% \cventry{2010--2015}{BSc Primary School Teacher}{Avans Breda, NL}{}{}{}

\section{Languages}
\cvitem{Languages}{Dutch (C2), English (C2), Spanish (C1)}

\end{document}
