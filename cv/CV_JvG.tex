% set-up
\documentclass[a4paper, 10pt, final]{moderncv}
\usepackage[scale=0.75, top=3cm, bottom=3cm, left=2.5cm, right=2.5cm]{geometry}
\moderncvstyle{casual} 
\moderncvcolor{blue}
\usepackage[utf8]{inputenc}
\usepackage[english]{babel}
\nopagenumbers{} 

% personal
\firstname{Jorrit}
\familyname{van Gils}
\title{Computer Vision Researcher}
\mobile{(+31) 6 27 57 22 61}
\email{vangilsjorrit@gmail.com}
% \address{Jan Oomenstraat 5, Bavel, NL}
\homepage{www.jorritvangils.com/}
\photo[64pt][0.4pt]{Jorrit.jpg} 
\social[linkedin]{jorritvangils}
\social[github]{JorritvanGils}
% \social[twitter]{jorritvangils}

\begin{document}

\makecvtitle

\section{Summary}
I’m a Computer Vision researcher at Wageningen University, advised by dr. Gert Kootstra dr. Gert Kootstra (Computer vision) and dr.ir Patrick Jansen (Ecology). 
My research focusses on extracting wildlife behaviour from camera trap images using machine and deep learning algorithms. 
Extensive experience in Python, JavaScript, Vue.js, and TensorFlow throught my role at object detection platform BOX21 and through annual 
participation in Kaggle-hosted classification competitions.

\section{Work Experience}
\cventry{2021--Present}{Developer Computer Vision Platform}{BOX21}{}{}{Developing and maintaining frontend-, backend systems and detection models.}
\cventry{2023}{Competitor Fathomnet competition}{Kaggle}{}{}{Detecting and classifying marine species in underwater images with Sean Nachtrab}
\cventry{2022}{Thesis Wageningen University}{}{}{}{Wildlife action recognition in camera-trap photographs using yolov5 and pose estimation}
% \cventry{2022}{Deep learning project}{}{}{}{Land-use classification using AlexNet on the UCM satellite dataset with Lars ter Kate}
\cventry{2022}{Minor Artificial Intelligence Wageningen University}{}{}{}{Programming in Python, Machine Learning, Deep learning}
\cventry{2022}{Competitor iWildCam competition}{Kaggle}{}{}{Counting the number of animals in a sequence of images.}

\section{Education}
\cventry{2019--2022}{MSc Forest and Nature Conservation}{Wageningen University, NL}{}{}{}
\cventry{2016--2019}{BSc Secondary School Teacher Biology}{Fontys Tilburg, NL}{}{}{}
% \cventry{2010--2015}{BSc Primary School Teacher}{Avans Breda, NL}{}{}{}

\section{Courses}
\cventry{2021}{Practical Deep Learning}{FastAi}{}{}{}
\cventry{2022}{Deep Neural Networks with Pytorch}{Coursera}{}{}{}
\cventry{2022}{Python Flask and SQL}{Udemy}{}{}{}
\cventry{2023}{Docker}{Coursera}{}{}{}
% \cventry{Jun 2022}{HTML and CSS}{Udemy}{}{}{}

\section{Computer skills}
\cvdoubleitem{Programming Languages}{Python, R}
  {Web Development}{HTML, JavaScript, Vue.js, SQL, Nginx, RabbitMQ}
\cvdoubleitem{Packages \& Software}{PyTorch, TensorFlow, WandB, Docker, GitHub, Linux}
  {Architectures}{YOLOv5, DeepLabCut, mmdetection, AlexNet, Query2Label}


\section{Languages}
\cvlanguage{Dutch}{C2}{}
\cvlanguage{English}{C1}{}
\cvlanguage{Spanish}{B2}{}
% \cvlanguage{German}{A2}{}
% \cvlanguage{French}{A1}{}

% \section{Awards}
% \cvline{Key Achievements}{
%   My project on animal behaviour recognition reached the Dutch newspaper. \\
%   \href{https://ibb.co/mBL0S5R}{\includegraphics[width=2cm]{image-icon.png}}
% }

% \section{Awards}
% \cvitem{Key Achievements}{
%   My project on animal behavior recognition reached the Dutch newspaper. \\
%   \href{https://example.com/link-to-image}{\includegraphics[width=2cm]{image-icon.png}}
% }


\end{document}
